\documentclass[a4paper, 12pt, openany]{book} %chose the paper size and font size. Openany ensures that all all chapters and similar may begin at any page, not only odd pages. For the introductory pages and appendices we want openany, but for chapter pages in the main content we want chapters to begin only on odd pages (right hand side). The book class ensures that the margins are automatically adjusted such that left hand pages are slightly moved to the left and vice versa at the right, which makes the thesis very readable and good looking when printed in bound book format.
\usepackage[utf8]{inputenc} %to manage special characters
\usepackage[T1]{fontenc} %to manage special characters
\usepackage[Bjarne]{fncychap} %fancy chapter style (many more available, like Sonny or Lenny etc.)
\usepackage{fancyhdr} %to customize the headers
\usepackage[lmargin=1.5in, rmargin=1in, tmargin=1in, bmargin=1in]{geometry} %sets the margins for the pages
\setcounter{tocdepth}{2} %table of contents number depth for subsections (2 = x.x.x)
\setcounter{secnumdepth}{4} %numbering depth for headers for subsections in the text(4 = x.x.x.x)
\usepackage{url} %to include urls
\usepackage{listings} %include this if you want to include code in the thesis
\usepackage{amsmath,amssymb} %mathematical package
\usepackage{siunitx} %includes SI-units
\usepackage[bf]{caption} %makes float captions bold
\usepackage{array, booktabs} %to make better tables
\usepackage{graphicx} %to include graphics
\usepackage{float} %to include floats
\usepackage[export]{adjustbox} %to adjust floats
\usepackage{subfig} %to include subfigures
\usepackage{chngcntr} %will make it possible to change the counter for tables, figures etc. such as below
\counterwithin{figure}{section} %change counter for figures within sections (also possible to choose for each chapter
\counterwithin{table}{section} %change counter for tables within sections
\usepackage{color, xcolor} %edit e.g. text colors

\usepackage[backend = biber,
            style = numeric,
            date = long,     % Long: 24th Mar. 1997 | Short: 24/03/1997
            sorting = none,
            maxcitenames = 3,   % max names to include before et. al.
            ]{biblatex} %customize the look of your citations and bibliography
\addbibresource{bibliography.bib} %declare the bibliography resource
\usepackage{comment} %to be able to comment out sections in the .tex files
\usepackage{afterpage} %to customize page commands such as below
\newcommand\myemptypage{
    \null
    \thispagestyle{empty}
    \addtocounter{page}{-1}
    \newpage
    } %sets new page command to insert an empty page without adding to the page counter or having a page number




\begin{document}
%%%%%%%%%%%%%%%%%%%%%%%%%%%%%%%%%%%%%%%%%%%%%%%%%%%%%%%%
%\begin{comment}
% The title page:
% For NTNU students this page will be generated automatically when submitting your paper, and should not be included in the final file from Latex. Delete or comment out the title page setup. The final report should then start with the first page being the abstract. I have included a title page here so it is possible to see how it may look like, and for those who does not get an automatically generated title page. Of course you will need to change the names and titles etc. to your case.

%the title page should be an odd page (right hand side)

\begin{titlepage}
\newgeometry{left=1.6in, right=2in}
\vspace*{1.5cm}

\noindent  \textcolor{gray}{\large Nina Salvesen} \\
\vspace{1cm}

\noindent \textbf{\Large The title of your master's thesis should be written here} \\
\vspace{0.5cm}

\noindent {\large Any undertitle is written here} \\



\vspace{7cm}
\noindent Master's thesis in Physics and Mathematics \\
Supervisor: Supervisor Name \\
Co-supervisor: Co-supervisor Name \\
June 2022 \\

\vspace{0.2cm}
\noindent Norwegian University of Science and Technology \\
Faculty of Natural Sciences \\
Department of Physics \\

\begin{figure}[h]
    \includegraphics[width=0.28\textwidth]{Figures/ntnu_basic.png}
\end{figure}
\end{titlepage}
\restoregeometry
\myemptypage %empty page such that the abstract starts at the first right hand side after the title page
%\end{comment}
%%%%%%%%%%%%%%%%%%%%%%%%%%%%%%%%%%%%%%%%%%%%%%%%%%%%%%%%

% The pre-chapters
\chapter*{Abstract} %pre-chapters should not be numbered, hence the "*"
\pagenumbering{roman} %introductory pages should be roman
\setcounter{page}{1}
\addcontentsline{toc}{chapter}{\protect\numberline{}Abstract} %add the chapter to the table of contents, this is not automatically added when creating unnumbered chapters (*). Add it in a chapter style, and keep all chapters on the same numberline indent regardless of number or not on the chapter

Write an abstract/summary of your thesis, and state your main findings here. \\

\noindent A summary should be included in both English and any second language, if this is applicable, regardless if the thesis is written in English or in your preferred language. These should be on separate pages, the English version first.







 %insert the chapter text from the files

\chapter*{Preface}
\addcontentsline{toc}{chapter}{\protect\numberline{}Preface} 

Write the preface of your thesis here. \\

\noindent You may include acknowledgements and thanks as part of your preface on this page, or you may add it as a new chapter after the preface.

\tableofcontents
\addcontentsline{toc}{chapter}{\protect\numberline{}Contents}

%add to table of contents list of figures and tables, and insert list of figures and tables
\addcontentsline{toc}{chapter}{\protect\numberline{}\listfigurename}
\listoffigures
\addcontentsline{toc}{chapter}{\protect\numberline{}\listtablename}
\listoftables


\chapter*{Abbreviations}
\addcontentsline{toc}{chapter}{\protect\numberline{}Abbreviations}
% Put in your abbreviations here

List of all abbreviations in alphabetic order:

\begin{itemize}
    \item \textbf{EDA} Exploratory Data Analysis
    \item \textbf{GNNS} Global Navigation Satellite System
    \item \textbf{Mamsl} meter above mean sea level
    \item \textbf{NTNU} Norwegian University of Science and Technology
    \item \textbf{PCA} Principal Component Analysis
    
\end{itemize}
\newpage
\myemptypage
%add an empty non-counted page by the command below in order to get the first chapter on the left hand side, if needed (check your page number so that the first chapter is on an odd page)


%%%%%%%%%%%%%%%%%%%%%%%%%%%%%%%%%%%%%%%%%%%%%%%%%%%%%%%%
%Customize the layout of the main content of your thesis

\pagestyle{fancy} %set customized page style for header
\fancyhf{} %clear header and footer fields
\renewcommand{\headrulewidth}{0pt} %set to no rule
\fancyhead[LE, RO]{\thepage} %set the page number at left for even, right for odd pages
\fancyhead[RE, LO]{\leftmark} %set the chapter name at right for even, left for odd pages
%is is possible to design the header with the chapter as you wish, e.q. only the chapter or only the name, all lowercase instead etc.
%you could also design the footer if you wish, for example:
%\fancyfoot[LE, RO]{\thepage}
\setlength{\headheight}{14.49998pt} %set the header height


%%%%%%%%%%%%%%%%%%%%%%%%%%%%%%%%%%%%%%%%%%%%%%%%%%%%%%%%
%main content 

\pagenumbering{arabic}
\chapter{Introduction}
\input{Chapters/03Introduction}
\cleardoublepage
%the cleardoublepage command ensures that the next text page is on the right-hand side (odd page) and produces a blank page if necessary to achieve that, as all chapters should begin on the right hand side


\chapter{Theory}
%equations, bib. \\

\section{Equations}

A simple equation can be included as: \\

\begin{equation}
        r = 2\pi^{2}
        \label{eqn:simple_eqn}
\end{equation} \\


\noindent The text below shows an example of how to align equations on the equal sign, with only one reference for both. This may be useful for when they are linked and are actually only one equation but splitting them up makes it more readable. \\

\begin{equation}
\begin{aligned}
        a &= \sin^{2}(\Delta\phi/2) + \cos(\phi_{1})\cdot\cos(\phi_{2})\cdot\sin^{2}(\Delta\lambda/2)\\
        d &= 2R\cdot\arcsin(\sqrt{a})
\end{aligned}
\label{eqn:haversine}
\end{equation} \\

\noindent The whole equation can be referenced as "equation \eqref{eqn:haversine}", here showing the Haversine formula. One may also align sub-equations such that they are numbered the same but have a letter differentiating them as shown below. This can be used when they are linked, but you will need to reference both individual parts.

\begin{subequations}
\begin{align}
    SSD 
        & \quad = \sum_{i=1}^{n} (\vec{x_{i}}-\vec{\mu_{q}})^{2} \label{eq:ssd}\\[15pt] %creates more space between the sub-equations
    SSE 
        & \quad = \sum_{q=1}^{k} \delta_{rq} SSD \label{eq:sse}
\end{align}
\label{eqn:subeqn}
\end{subequations} \\

\noindent These equations can be refrenced by their spesific sub-equation as "equation \eqref{eq:ssd}", or by the whole group as "equations \eqref{eqn:subeqn}". The "double backslashes" in the .tex creates line spaces and gives more room around the equations and paragraphs. Use them as you think feels right. 



\section{Tables and footnotes}

Here is an example of both a regular table with data and a table with split headers, for scientific usage. Do not use horizontal/vertical rulers between the data, or encase the table with rulers. \\

\begin{table}[ht!]
\centering
    \begin{tabular}{ m{3cm} m{2.5cm} m{2.5cm} m{2.5cm} m{2cm} } 
    \toprule
    \toprule
    \textbf{Statistic} & \textbf{Velocity} & \textbf{Altitude} & \textbf{1/Angle} & \textbf{Temp.} \\
    \midrule
    Mean    & 122.68    & 240.98   & 93.75     & 13.95 \\[1.3ex]
    Std     & 224.51    & 145.88   & 60.39     & 4.44  \\[1.3ex]
    Q1      & 28.00     & 111.60   & 34.15     & 10.60 \\[1.3ex]
    Median  & 63.00     & 223.20   & 99.59     & 13.30 \\[1.3ex]
    Q3      & 137.00    & 359.10   & 151.99    & 16.70 \\[1.3ex]
    Min     & 0.00      & 1.00     & 0.00      & 3.30  \\[1.3ex]
    Max     & 14519.00  & 616.70   & 180.00    & 32.10 \\[1.3ex]
    \bottomrule
    \bottomrule
    \end{tabular}
% the square brackets after \caption gives the table a proper title in the list of tables, instead of just inserting the beginning of the table caption
\caption[Dynamic feature statistics with outliers]{Table of dynamic feature statistics where outliers are included, for all data points. Velocity is given in \textit{m/h}, the altitude in \textit{mamsl}, the inverse trajectory angle in 1/degrees, and temperature in degrees Celsius.}
\label{table:stat_fliers}
\end{table}


\begin{table}[ht!]
\centering
    \begin{tabular}{ m{3cm} m{5cm} m{3cm} } 
    \toprule
    \toprule
    \textbf{Area 1} & \textbf{Start date} & \textbf{End date} \\
    \midrule
    2018    & 03.06    & 29.06                       \\[1.3ex]
    2019    & 03.06    & 03.07 or 31.08\footnotemark \\[1.3ex]
    2020    & 03.06    & 05.09                       \\[1.3ex]
    \midrule
    \textbf{Area 2} & \textbf{Start date (farm 1/2)} & \textbf{End date} \\
    \midrule
    2012    & 09.06            & 07.09               \\[1.3ex]
    2013    & 23.06 / 15.06    & 25.08               \\[1.3ex]
    2014    & 05.06 / 25.06    & 10.09               \\[1.3ex]
    2015    & 13.06 / 03.07    & 06.09               \\[1.3ex]
    2016    & 17.06            & 22.07               \\[1.3ex]
    \bottomrule
    \bottomrule
    \end{tabular}
\caption[Selected time ranges for all data]{Selected time ranges for the data in all areas and all years.}
\label{table:time_ranges}
\end{table}

\footnotetext{A footnote explaining something.}



\section{A single figure}

Figure \ref{fig:latlong} is included as an example. The square brackets before the caption description contains the title of the figure, which is what will be written in the list of figures. This should be short and concise. The same layout applies to tables and other floats. Remember to change the title as well as the caption if you are copying these examples. In the list of figures and tables all the different floats will be grouped together by chapter. Remember to always reference all your figures and tables in the text at least once.

\begin{figure}[H]
  \centering
  \includegraphics[width=0.9\textwidth]{Figures/latlong.png}
  \caption[Illustration of latitude and longitude]{Illustration of the earth, and how latitudes and longitudes are calculated with respect to the equator and the prime meridian.}
  \label{fig:latlong}
\end{figure}


\section{Citations}

Here are some examples on how to reference a source (where none is relevant to the text but just for illustration purposes only). One may cite a single reference by calling \cite{wolves_of_mount_mckinley}, or several in the same bracket by calling \cite{machine_learning, clustering_impossibility} when they are all related to the same statement. There are many different styles on how to cite, and how the layout and order of your citation style is presented. This is my favorite, as I find it neat and tidy \cite{sheep}. It will show up in order of appearance in the references section.
\cleardoublepage


\chapter{Methods}

Include the complete description of the methods used in your research here. \\

\noindent Below is an example of how subsectioning works. The sections and subsections will be included in the table of contents, while subsubsections will not be in the table of contents but still have their own title in the text.

\section{Section one}

\subsection{Subsection one}

\subsubsection{Subsubsection one}

\subsubsection{Subsubsection Two}

\subsection{Subsection Two}

\section{Section two}


\cleardoublepage


\chapter{Results}

\section{More figures}

This section includes some examples of different types of figures to include. A simple single figure is shown in figure \ref{fig:trajectory_angle}, while figure \ref{fig:cyclic} shows how three subfigures can be included together. Remember to change both the caption and the title in the square brackets before the caption, which will show up in the list of figures. \\

\noindent page fill page fill page fill page fill page fill page fill page fill page fill page fill page fill page fill page fill page fill page fill page fill page fill page fill page fill page fill page fill page fill page fill page fill page fill page fill page fill page fill page fill page fill page fill page fill page fill page fill page fill page fill page fill page fill page fill page fill page fill page fill page fill page fill page fill page fill page fill.

\begin{figure}[H]
  \centering
  \includegraphics[width=1\textwidth]{Figures/trajectory angle.png}
  \caption[Trajectory angle]{The trajectory angle found for the trajectory ABC between the three points A, B and C. The two different C-points show the angle gotten for relatively unchanged directional trajectory with C', and opposite directional trajectory with C''.}
  \label{fig:trajectory_angle}
\end{figure}



\begin{figure}[H]
  \centering
  \subfloat[Time as a linear variable.]{\includegraphics[width=0.75\textwidth]{Figures/diurnal_values_linear.png}\label{fig:diurnal_lin}}
  \hfill
  \subfloat[Time represented as pairs of sine and cosine.]{\includegraphics[width=0.75\textwidth]{Figures/diurnal_values_sine.png}\label{fig:diurnal_trig}}
  \hfill
  \subfloat[Time as a cyclic feature.]{\includegraphics[width=0.6\textwidth]{Figures/diurnal_values_circle.png}\label{fig:diurnal_cyclic}}
  
  \caption[Time as a cyclic feature]{Figure (a) shows the time variable in the data for the first 100 rows using linear time as minutes past midnight, figure (b) shows the trigonometric time representation for one cycle where each time point has a unique sine and cosine-pair value, and figure (c) shows time as a cyclic feature with the trigonometric pairs.}
\label{fig:cyclic}
\end{figure}
\cleardoublepage


\chapter{Discussion}

Discuss your results here.

\section{Future work}

Include a section about what should or could be done in future research, or explain any recommended next steps based on the results you got. This should be the last section in the discussion.
\cleardoublepage


\chapter{Conclusions}

Give a concise summary of your research and finding here, and include a short summary of any future work as well.
\cleardoublepage


\addcontentsline{toc}{chapter}{\protect\numberline{}References}
\printbibliography[title={References}] %you may change the title in the toc here if you want
\cleardoublepage


\chapter*{\LARGE \textbf{Appendices}}
\fancyhf{} %clear the header, it should be empty for the appendices
\renewcommand{\headrulewidth}{0pt} %no rule
\fancyfoot[C]{\thepage} %set the page numbers in the center of the footer instead 

%it is possible to set a different page numbering style for the appendix, but I personally just continued with the same page numbering as the main content as I find that more tidy
%\pagenumbering{roman}
%\setcounter{page}{1}
\addcontentsline{toc}{chapter}{\protect\numberline{}Appendices:}
\appendix


\chapter*{A - Github repository}
\addcontentsline{toc}{chapter}{\protect\numberline{}A - Github repository} 

All code and latex-files used in this document are included in the Github repository linked below. Further explanations are given in the readme-file. 


\subsection*{Github repository link}
\begin{itemize}
    \item \url{https://github.com/ninasalvesen/thesis_latex_template}
\end{itemize}


%%%%%%%%%%%%%%%%%%%%%%%%%%%%%%%%%%%%%%%%%%%%%%%%%%%%%%%%


\chapter*{B - Sidenote statistics}
\addcontentsline{toc}{chapter}{\protect\numberline{}B - Sidenote statistics} 

%For included tables and figures renew the numbering such that they are numbered by the appendix they are attached to and not to the conclusion chapter
\renewcommand{\thefigure}{B.\arabic{figure}}
\setcounter{figure}{0}
\renewcommand{\thetable}{B.\arabic{table}}
\setcounter{table}{0}


\section*{\large{B1 - Some random table}}
\vspace*{1cm}

Remember to only include one thing per page in the appendices.

\begin{table}[ht!]
\centering
    \begin{tabular}{ m{4cm} m{2.5cm} m{2.5cm} m{2.5cm} } 
    \toprule
    \toprule
    \textbf{Statistic} & \textbf{One} & \textbf{Two}  \\
    \midrule
    Count   & 387317    & 283960    \\[1.3ex]
    Mean    & 130.66    & 134.18    \\[1.3ex]
    Std     & 248.09    & 230.32    \\[1.3ex]
    Q1      & 31.00     & 21.00     \\[1.3ex]
    Median  & 67.00     & 63.00     \\[1.3ex]
    Q3      & 142.00    & 159.00    \\[1.3ex]
    Min     & 0.00      & 0.00      \\[1.3ex]
    Max     & 14519.00  & 14253.00  \\[1.3ex]
    \bottomrule
    \bottomrule
    \end{tabular}
\caption[Statistics on something]{Table of statistics on some sidenote data.}
\end{table}


% Page without title but section title:
\newpage
\section*{\large{B2 - Some other random table}}
\vspace*{1cm}

\begin{table}[ht!]
\centering
    \begin{tabular}{ m{4cm} m{2.5cm} m{2.5cm} m{2.5cm} } 
    \toprule
    \toprule
    \textbf{Statistic} & \textbf{Three} & \textbf{Four}  \\
    \midrule
    Count   & 387317    & 283960    \\[1.3ex]
    Mean    & 130.66    & 134.18    \\[1.3ex]
    Std     & 248.09    & 230.32    \\[1.3ex]
    Q1      & 31.00     & 21.00     \\[1.3ex]
    Median  & 67.00     & 63.00     \\[1.3ex]
    Q3      & 142.00    & 159.00    \\[1.3ex]
    Min     & 0.00      & 0.00      \\[1.3ex]
    Max     & 14519.00  & 14253.00  \\[1.3ex]
    \bottomrule
    \bottomrule
    \end{tabular}
\caption[Statistics on something else]{Table of statistics on some other sidenote data.}
\end{table}



\newpage
\section*{\large{B3 - Some random figure}}
\vspace*{1cm}

\begin{figure}[H]
  \centering
  \subfloat[Data set sizes for data X.]
  {\includegraphics[width=1\textwidth]{Figures/dataset_X.png}}
  \hfill
  \subfloat[Data set sizes for data Y.]
  {\includegraphics[width=1\textwidth]{Figures/dataset_Y.png}}
  \caption[Data set sizes]{The figures show the data set sizes of X and Y and the proposed cut-off threshold at 10 $\%$ of the mean set size.}
\end{figure}



%%%%%%%%%%%%%%%%%%%%%%%%%%%%%%%%%%%%%%%%%%%%%%%%%%%%%%%%


\end{document}
